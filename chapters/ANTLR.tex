ANTLR e' un parser generator che genera un parser LL top down multi linguaggio. A partire da una grammatica possiamo generare il parser nel linguaggio che vogliamo e ci genera un runtime nel linguaggio che vogliamo per fare funzionare il parser.

La roba interessante e' che ci sono un sacco di grammatiche pronte per ANTLR.

Abbiamo due modi per usare questo strumento:
\begin{itemize}
    \item Uso diretto di ANTLR, leggendo il manuale e capire come funziona per scrivere il codice python
    \item Uso mediato da Liblet, significa che il prof ha scritto un po' di codice in Liblet che serve per importare roba di Java in maniera dinamica
\end{itemize}

Nel progetto e' una delle cose che possiamo scegliere.

\section{Uso diretto di ANTLR}

La cosa principale di questo strumento e' che ci permette di specificare la grammatica context free che vogliamo usare per produrre il parser tramite file o stringa. Questa grammatica e' fatta cosi', definiamo la grammatica, poi ci sono una serie di regole di tokenizzazzione scritte in maiuscolo alla cui destra c'e' una espressione regolare. Le regole della grammatica (di parsing) iniziano in minuscolo, sono date da un simbolo di tokenizzazzione e poi si possono mettere delle costanti.
Alcuni token possono essere scartati dalla parte di parsing.

\begin{lstlisting}
    
\end{lstlisting}

Adesso il punto cruciale e' utilizzare un tool Java che generi codice Python. Lui genera 4 file:
\begin{itemize}
    \item lexer
    \item parser
    \item visitor: interfaccia che permette di visitare il tree generato dal parser
    \item listener: interfaccia
\end{itemize}

%Esempio di uso

Uso del lexer, possiamo vedere i passi del lexer, il token ci fa vedere anche i caratteri del sorgente che rappresentano il token:


Abbiamo visto un esempio piu' complesso di una grammatica di espressioni ed assegnamenti. Vediamo ora come usare il paser, tendenzialmente i parser hanno due modalita' di funzionamento:
dato che l'albero e' una struttura ricorsiva ci sono due modi per poterne fruire. Ognuno di questi nodi e' relativo ad una delle regole di parsing, i figli sono i token.
\begin{itemize}
    \item listener: ha due metodi per ogni regola, entro ed esco. Poi posso invocare un processo di visita che invoca il listener secondo l'ordine che mi aspetto.
    \item visitor: me lo scrivo io, mi scrivo una funzione per ciascuna regola che sono quelle usate per fare la visita.
\end{itemize}

Quello che faremo noi a lezione e' usare visitor, perche' a lezione usando LIBLET ci dara' un albero e noi ci scriveremo sempre le visite dell'albero.
Cosi' e' come useremo il listener, creo una classe che implementa il listener, poi creo l'istanza e lo passo al walker di liblet che lo usa per fare la visita:
\begin{lstlisting}

\end{lstlisting}

Nel caso del visitor ho sempre la mia classe che implementa il vistor, mi implemento dentro la visita e poi uso la classe per fare la visita che mi sono scritto:
\begin{lstlisting}

\end{lstlisting}

Fino ad ora non ci siamo mai occupati degli errori, ma in caso di ANTLR si possono specificare degli ascoltatori di errore che hanno un metodo che va invocato ogni volta che lui vede un errore nel lexing o nel parser:
\begin{lstlisting}

\end{lstlisting}

\section{Uso mediato da Liblet}
Il modulo ANTLR dentro Liblet ci permette di prendere una grammatica e mi da un oggetto che ha delle competenze.
In questa grammatica ci sono delle etichette di ANTLR che ci servono a distinguere le alternative di una regola che ha delle alternative che e' comodo perche' se avviene un match di stat non sapremmo cosa abbiamo matchato:
\begin{lstlisting}

\end{lstlisting}

Dato l'oggetto possiamo fargli fare diverse cose, inizialmente vediamo i tokens, basta un metodo in questo caso:
\begin{lstlisting}

\end{lstlisting}

Posso avere anche il contesto, che mi da dove partire per il visitor, posso dargli anche il token da cui partire:
\begin{lstlisting}


\end{lstlisting}

L'albero di parsing possiamo produrlo con un metodo tree che produce un tree di liblet con dentro le espressioni:
\begin{lstlisting}

\end{lstlisting}

ANTLR accetta anche grammatiche malate (ricorsivita', precedenze...) c'e' pero' una malattia che non puo' essere riconosciuta a tempo di analisi, la ambiguita'. Si puo' accendere una spia che ci dice se il parsing e' ambiguo, il suggerimento e' che quando si progetta una nuova grammatica si preparano dei testcase e vedo se il diagnostici mi dice se e' la grammatica e' ambigua. Va acceso il diag = True nel context:
\begin{lstlisting}

\end{lstlisting}

%manca tutta la prima parte della lezione 16, sono arrivato tardi

\subsection{Grammatiche note}
Possiamo guardare alcune grammatiche note che possono essere utili, ad esempio c'e' una grammatica per l'aritmetica, bella ma l'associativita' a destra la sbaglia.
Un'altra grammatica carina e CALCULATOR, che parsa le espressioni e permette l'invocazione di funzioni, come il seno o coseno.

Un'altra grammatica super utile, che non viene dal repository standard, ma dal libro di antlr e' la grammatica del linguaggio Cymbol che e' una specie di linguaggetto di programmazione che ha l'invocazione a funzione e gli statement.

La fine di questa cosa e' che e' molto potente, e se incontriamo errori guardiamo in documentazione perche' ci sono alcune porcherie.

\subsection{Ancora su listener e visitor}
Ora ci assegniamo due compiti ed useremo listener e visitor per fare due cose diverse:
\begin{itemize}
    \item Numerare le righe
    \item Interpretare un linguaggio
\end{itemize}

Il linguaggio a cui faremo riferimento e' il linguaggio aritmetico con nomi di variabili ed assegnamenti. Una sequenza lineare di istruzioni in cui le istruzioni sono assegnamenti, stampe o espressioni.

%grammatica e programma

Vogliamo contare gli assegnamenti, lo faremo con un listener, sappiamo che ANTLR ci da' un listener che e' un interfaccia da estendere, noi lo estendiamo e implementiamo i metodi che ci interessano. In questo caso ci interessa il numero di assegnamenti, quindi implementiamo il metodo che ci interessa. Poi creiamo un oggetto del listener e lo passiamo al walker di liblet.

\begin{lstlisting}

\end{lstlisting}

Ora vogliamo interpretare il programma, un compito cosi' sofisiticato spesso si realizza piu' facilmente con un visitor, perche' quando scrivo un linguaggio ho in testa per ogni nodo del linguaggio cosa voglio fare, in particolare noi vogliamo che per ogni nodo restituisca il valore del nodo, questo lo puo' fare facilmente se fa una visita in post-ordine conoscendo i valori dei figli.

In questo caso abbiamo solo variabili locali, quindi la cosa piu' facile da fare e' tenere una memoria che e' un dizionario di python, se vediamo un assegnamento prendiamo il nome della variabile, facciamo lo scarico ricorsivo chiamando expr e salviamo tutto in memoria. Se vediamo una variabile, facciamo un lookup in memoria e restituiamo il valore. Se vediamo un numero lo restituiamo direttamente, da notare che quando restituiamo l'intero e' l'unico passo interpretativo perche' stiamo trasformando testo in intero. Abbiamo un espressione che fa la print, non ritorna niente ma fa una print. Dopo di che c'e' la parte di associazione, di nuovo sono divise perche' sono cose divise, in ogni caso visito tramite scarico ricorsivo il lato sinistro e destro e poi guardo il tipo dell'operatore, la visita delle parentesi e' un trucco sintattico che serve ad indicare diverse precedenze, quindi dal punto di vista interpretativo non serve a niente e si puo' solo mangiare.
%perche' due metodi per la somma e la moltiplicazione? 
\begin{lstlisting}

\end{lstlisting}

Estendere questo parser su linguaggi piu' difficili e' immediato, la logica e' sempre la stessa, basta estendere le regole di parsing e le regole di interpretazione. Potrebbe essere un buon esempio. Il trucco cruciale e' che il visitatore e' ricorsivo quindi posso quando ho un dubbio su che cosa fare posso usare la ricorsione, per valutare un espressione valuto ricorsivamente i figli e poi faccio l'operazione.
Il problema di questa cosa e' che la pila di chiamate ricorsive puo' esplodere e non e' facile da tenere sotto controllo, soprattutto se vogliamo dare degli errori durante il parsing. Quindi abbiamo bisogno di fare una traduzione in cui rendo la ricorsione esplicita, significa liberarsi della ricorsione passando ad una struttura iterativa, significa avere una pila di chiamate esplicita invece della pila di record di attivazione, l'idea qui e' che tengo dei risultati parziali. Qui dobbiamo scegliere tra visitor che usa la ricorsione e listener in cui teniamo una pila esplicita per le chiamate.

L'idea in questo caso e' di tenere gli operandi sulla pila e quando siamo pronti a fare l'operazione togliamo gli operandi e mettiamo il risultato intermedio.
Quindi il listener avra' due strutture dati per funzionare, una memoria per le variabili ed una pila. Qui ho bisogno degli exit perche' voglio sapere cosa fare con quello che c'e' sulla pila dopo che ho guardato i figli. Se usciamo da un assegnamento significa che ho gia' valutato l'espressione e sulla pila c'e' il valore da assegnare, quindi assegniamo in memoria. Se usciamo da un id o un intero significa che valutiamo il valore e mettiamo sulla pila il valore. Quando usicamo da una divisione mi aspetto che i valori dei suoi operandi siano sulla pila messi dalla visita in post-ordine, quindi li tolgo dalla pila e metto il risultato della divisione sulla pila. Se usciamo da una somma invece mettiamo il risultato della somma sulla pila.

\begin{lstlisting}

\end{lstlisting}

\section{Dall'albero di parsing all'AST}
L'idea e' di partire da un albero di parsing che ha tanti artefatti, toglierli e raggiungere una forma compatta che e' un abstract syntax tree. Per farlo usiamo una funzione ricorsiva.

Abbiamo visto che in ANTLR si puo' mettere del codice direttamente nella grammatica, non lo faremo mai e' troppo doloroso, poi non ha troppo senso tenere legata la grammatica con l'implementazione del parser.

\subsection{Dispatch table}
Useremo una dispatch table per capire cosa fare quando troviamo una regola (nella table ci saranno le funzioni ricorsive). Inizialmente diciamoche quando troviamo una regola e c'e' nella dispatch table chiamo la sua regola, se non la trovo metto tutti in un catchall:
\begin{lstlisting}
    
\end{lstlisting}

Devo riempire la tabella, in questo caso definiamo le funzioni che ci servono quando vediamo gli operatori, da notare che come scarico ricorsivo non usano se stesse ma chiamano la funzione che usa la dispatch table, poi popolo la table:
\begin{lstlisting}
    
\end{lstlisting}

Si e' visto che liblet implementa una classe TreeWalker che fa questo lavoro della funzione ricorsiva, possiamo dirgli dato un albero che attributo considerare, registrare delle funzioni che usa quando incontra dei simboli (con un annotazione o registrarle esplicitamente), inoltre permette anche di avere un catchall (ne esistono di tanti tipi).

Vediamo un esempio in cui eseguiamo il programma partendo da un ATS, notiamo che occupandoci degli atomi abbiamo solo const e var, non abbiamo bisogno di castare ad Int perche' lo abbiamo gia' fatto nella costruzione dell ATS quando mettevamo dentro il valore di una costante.

\begin{lstlisting}
    
\end{lstlisting}

\section{Trasformare}
Vediamo alcuni esperimenti di trasformazione, in particolare inizialmente vogliamo trasformare un documento JSON in un documento HTML usando ANTLR.

Quello che facciamo noi e' modularizzare il processo, partiamo generando l'AST, si fa con un tree walker ricorsivo che ha la tabella di dispatch che discrimina i nodi in base al nome.

\subsection{Trasformare HTML in matrici}
L'idea e' che ho acceduto ad una pagina HTML con dentro delle tabelle e io voglio farmi una matrice che rappresenta i dati, e' una trasformazione di dominio.

Per farlo partiamo con il definirci una nostra grammatica invece di usare quella completa assumendo per semplicita' che dentro i td possano esserci solo interi.

Il processo e' sempre lo stesso, arriviamo all'AST e qui la cosa che cambia e' che per fare la trasformazione usiamo una struttura dati esterna, uno stack, quando vediamo i dati li pushiamo dentro lo stack e quando vediamo una row togliamo i figli e li mettiamo dentro un array, alla fine dello stack avremo un array di array che rappresenta la matrice.

Abbiamo visto anche che per costruire la tabella potremmo pensare di decorare i nodi con il numero di riga e colonna mettendo nella radice il numero di righe e colonne, cosi' posso poi precomputare una matrice di dimensioni fisse piena di None e poi riempirla visitando l'AST.

\section{Traspilazione}
L'idea e' quella di prendere un linguaggio e trasformarlo in un altro linguaggio. Questa cosa ha origini storiche perche' ad esempio il C non veniva compilato ma traspilato in Fortran. Ci sono anche logiche di portabilita' e di ottimizzazione. La portabilita' e' moderna, uno dei linguaggi dominanti nel web e' JavaScript, il problema di questo linguaggio e' che non e' standardizzato e non coinvolge tutti i browser. Ragione per cui e' nata una famiglia di traspilatori che prendono un JS recente e lo trasformandono in un JS piu' vecchio che gira su tutti i browser. Un esempio e' Babel, che e' un traspilatore di JS, ma ce ne sono anche altri. Se esistono costrutti che nelle vecchie versioni non esistono si usano gli SHIM. E' nato anche typescript, che non esiste nei browser, ma viene traspilato in JS.

Noi partiamo da un linguaggio semplice SimpleLang, prima lo traspiliamo in JavaScript (facendo un text to text trsaplilazione). L'altro passo che faremo e' quello di sfruttare il fatto che alcuni linguaggi hanno una rappresentazione di alto livello, ma che e' un po' piu' snella di un testo (non richiede ulteriori processi di parsing) che puo' essere complata dalla VM e generata senza troppo sforzo, quindi tradurremo questo linguaggio verso Python facendo una traduzione da testo verso l'AST di python (AST = Abstract Syntax Tree).

Dobbiamo decidere come fare l'IO, perche' ha senso compilare un programma solo se c'e' un input. Noi per farla facile cercemo di provvedere modi banali per dare l'input ai nostri programmi, l'interprete preleva delle cose dal mondo esterno e le inietta nel programma. Nel nostro caso assumiamo che ci siano tre variabili INPUT e per farla facile faremo che ha un solo OUTPUT. (dice che il progetto di quest'anno non avra' bisogno di librerie esterne).


Arrivati infondo alla traspilazione dobbiamo cercare di valorizzare la variabile INPUT, facciamo una funzione che cerca tutti gli input e per ciascuni di essi apre un prompt che prende l'input.

Vediamo ora come cercare di evitare il passo di parsing del linguaggio in cui arriviamo. Scriviamo il linguaggio in una forma in cui l'interprete di python sia in grado di interpretarlo direttamente, python perche' e' fornisce delle API per interagire con gli AST.
Si puo' usare eval per fare valutare una funzione e generare un parse tree. una volta che abbiamo un AST possiamo compilarlo con la funzione complie di python che vuole sapere cosa sta compilando e in che modalita'.
Per fare la traspilazione verso l'ATS di python dobbiamo visitare il nostro albero e trasformare i nostri nodi nei loro corrispondenti nodi di python.

Dopo siamo di nuovo nella situazione di dover valorizzare gli input, i programmi di python hanno un dizionario dove pesca le variabili locali, noi possiamo aggiungere con un dizionario di python le variabili che ci servono. Ora li stiamo mettendo a mano noi, potremmo anche pensare di mettere delle funzioni per valorizzare le variabili di input.

\section{Symbol table}
Assomiglia ad un dizionario, abbiamo due funzioni Bind e Look up con cui mettiamo dentro un simbolo e lo cerchiamo. Hanno anche la competenza di entrare ed uscire da un contesto (ENTER, EXIT) come contesto potremmo intendere classi o funzioni.

\subsection{Scope}
Nei linguaggi di programmazione moderni le variabili non solo visibili ovunque, ma hanno uno scope. Per farlo utilizzeremo delle opportune symbol table. Esiste uno scoping locale che serve a correggere uno scoping globale. Possiamo avere anche uno scope unico o distinto posso usare lo stesso simbolo in contesti diversi o meno (una variable che si chiama come una funzione).

\subsection{Implementazione symbol table}
Esistono svariate implementazioni, noi ne vedremo due.

Implementazione deterministica: e' la piu' facile ed adoperiamo una pila di coppie (simbolo, informazione), fare il bind di un simbolo significa fare il push della coppia simbolo informazione. Il lookup e' la ricerca in questa pila dall'alto verso il basso. Enter spingo nella pila qualcosa che sono sicuro di riconoscere tipo null, null, in modo tale che quando faccio exit posso fare una catena di pop fino a che non trovo il separatore, sostanzialmente metto un separatore per i contesti. E' un po' inefficiente perche' quando faccio un lookup devo scorrere tutta la pila, e anche l'exit non e' bellissimo.

Possiamo vedere anche la symbol table come albero di dizionario, avremo dei dizionari (chiave - valore) attaccati al loro papa'. Quindi in questo caso dobbiamo fare una ricerca al massimo lineare nel numero di scope, poi dentro i dizionari la ricerca e'  0(1). La cosa bella e' che quando faccio un enter creo un dizionario vuoto e lo metto come figlio del dizionario padre, quando faccio un exit lo tolgo. In questo caso non abbiamo bisogno di mettere i separatori, perche' il dizionario padre e' il separatore.

\section{Interpretazione}
Siamo arrivati al punto in cui possiamo cominciare a parlare di interpretazione, siccome l'albero della AST e' una struttura ricorsiva, possiamo fare una visita ricorsiva dell'albero e interpretare il linguaggio. Questa e' la cosa piu' semplice da fare.

Dopo di che aggiungeremo le funzioni e penseremo a come interpretare le funzioni, con l'idea che le funzioni siano rientranti, quando esco dalla funzione ritorno da dove l'ho chiamata, inoltre i parametri della funzione sono trattati come variabili locali nello scope della funzione, per fare questo useremo la pila dei record di attivazione (chiamato anche call stack).

Come esempio abbiamo sempre il SimpleLang in cui abbiamo introdotto anche il prodExpr. Vediamo come fare l'interprete ricorsivo, inchiodiamo una variabile globale e lo attacchiamo all'interprete, in python si possono incollare attributi agli oggetti, in questo caso lo facciamo con interpreter.GLOBAL\_MEMORY.
Il difetto dell'interpretazione ricorsiva e' il costo di esecuzione. Non abbiamo fatto robe molto innovative.

\subsection{Funzioni}
Vogliamo dare una nuova espressivita' al linguaggio, vogliamo dare un nome ad un pezzo di codice per eseguirlo piu' volte. Una funzione ha una collezione di parametri (formali) che operano dal punto di vista concettuale come variabili locali di quella funzione.
Dobbiamo definire due momenti:
\begin{itemize}
    \item Definizione di funzione: nome, parametri formali, corpo
    \item Invocazione di funzione: nome, parametri concreti (actual) che sono di fatto delle espressioni, potremmo avere il valore di ritorno potremmo quindi dividere le funzioni tra quelle che calcolano un valore a quelle che non lo fanno.
\end{itemize}

In quasi tutti i linguaggi un espressione costituisce uno statement e noi faremo cosi', inoltre ci occuperemo solamente di funzioni di primo livello, non ci occuperemo di chiusure.

Ci interessa ora vedere come usare la pila di chiamate.
Inanzitutto abbiamo deciso che un programma e' una sequenza di funzioni, decidiamo che ci sia una funzione che eseguiamo prima delle altre (main), dobbiamo inserire nella grammatica la definizione di funzione con i parametri formali e nell'invocazione delle espressioni possono esserci altre espressioni. Le espressioni diventano statement e ci mettiamo il return statement che e' una comoda convenzione per indicare il valore di ritorno della funzione, questo return e' opzionale se non c'e' return assumeremo che la funzione ritorni None.
%magari qui mettila la grammatica

Ora vediamo cosa fare quando visitiamo il nodo di dichiarazione di funzione, puo' succedere che ci siano i parametri formali o no, dobbiamo salvarci il nome della funzione e i parametri formali. Da stare attenti che nel caso in cui ci siano i parametri devo andare di due in due perche' devo saltare la virgola.

La chiamata e' sostanzialmente diversa, di nuovo posso avere gli argomenti o meno (anche in questo caso devo saltare la virgola). In questo caso devo creare un nuovo albero che chiama la funzione.

Adesso e' il momento di occuparci della chiamata. Dobbiamo predisporre un record di attivazione, che e' un dizionario in cui mettiamo i parametri formali e i valori dei parametri concreti. Poi dobbiamo fare l'enter della funzione, quindi dobbiamo mettere il record di attivazione nella pila dei record di attivazione, poi dobbiamo visitare il corpo della funzione. Alla fine della visita della funzione dobbiamo fare l'exit della funzione, quindi togliere il record di attivazione dalla pila e restituire il valore di ritorno. E' una pila perche' man mano che chiamo aggiungo record e quando ritorno faccio il pop del record di attivazione. Quindi valutiamo le espressioni dei parametri concreti, prepariamo il record di attivazione, facciamo il push del record di attivazione nella pila, passiamo il controllo, quando ho finito di fare faccio il pop del record di attivazione e restituisco il valore di ritorno.

Aggiungiamo un altro attributo globale ACTIVATION\_RECORDS che e' una pila. La modifica cruciale che invece di cercare i valori nella memoria globale li cerchiamo nella pila di attivazione.

L'interpretazione del programma diventa la raccolta dell'interpretazione delle funzioni, le raccogliamo in un dizionario FUNCTIONS che indicizziamo con il nome delle funzioni e dentro ci mettiamo gli alberi di parsing delle funzioni.
Interpretare il programma significa fare il visit del main, che come detto prima mettera' in pila il record di attivazione, a sua volta mettera' in pila i record di tutte le funzioni dopo main.

Il caso dell'istruzione return e' gestito aggiungendo una variable locale dal nome \_retval per conservare il valore restituito.

Nella chiamata per preparare il record di attivazione dobbiamo valutare i parametri concreti con le espressioni (che sono proprio numeri nel nostro caso), il record e' un dizionario di coppie.

Abbiamo anche visto come fare il ritorno forzato quando abbiamo un return, ad esempio in un for o in un if, quello che va fatto semplicemente e' che quando valutiamo quelle espressioni non andiamo giu' dritto come al solito, ma dopo il visit(stat) controllo che non ci sia \_retval nella memoria, se c'e' faccio return anche dal for o dall'if.

\section{Tipi}

Esistono errori untrapped o trapped. L'obiettivo e' avere un codice safe e well-behaved, per fare questo ci vengono in aiuto i tipi e tutte le tecniche che ci permettono di capire se il programma e' safe.

Alcuni linguaggi non hanno tipi ed altri sono tipizzati, in vario modo misura, nel senso che ogni istanza di variabile o funzione e' decorata con un tipo, questo si chiama tipi espliciti. Ci sono altri sistemi in cui i tipi non sono espliciti, ma sono impliciti, nel senso che il compilatore o l'interprete si occupa di dedurre il tipo della variabile o della funzione.
Quello che vogliamo vedere adesso e' se un programma e' well-typed, e' ragionevole pensare che se un programma e' ben tipato allora e' anche safe, in realta' questa implicazione e' delicata perche' oltre ad avere un sistema di tipi dobbiamo averlo sound (ben fatto). Attenzione che la soundness nei linguaggi comuni non e' garantita.

Se viviamo nel mondo dei linguaggi tipati espliciti possiamo usare il typechecker che verifica la well-behavior del programma in fase statica. Invece se stiamo nel mondo dei linguaggi untyped o typed implicit dobbiamo fare un controllo dinamico. Tipicamente si usano tecniche miste.

\subsection{Typechecking}

Quello che faremo e' decorare l'AST con delle symbol table, in modo tale che quando visitiamo l'AST possiamo vedere se il programma e' ben tipato o meno. La cosa bella e' che la symbol table e' una struttura dati che ci permette di tenere sotto controllo i tipi delle variabili e delle funzioni, quindi possiamo usarla per fare il type checking.

Siamo in un mondo con solo tipi primitivi, solo interi e stringhe. Possiamo fare tytte le espressioni binarie con entrambi gli operandi interi, confrontare due stringhe oppure sommare due stringhe (ottenendone la concatenazione) o moltiplicare una stringa per un intero (ottenendone la ripetizione). L'enforcing di queste regole non puo' appartenere all'analisi lessicale del linguaggio, ma deve appartenere ad un analisi successiva.